\documentclass{zebracorns} 

\title{Zebracorn Template}
\author{The Zebracorns}
\date{September 2019}

\begin{document}

\maketitle

\section{Introduction}
Overleaf has provided our robotics team, The Zebracorns, with a premium subscription to their software for sharing and editing LaTeX documents. We want to share our gratitude for their generosity, explain our process for writing, editing, and publishing whitepapers, and talk about the features of Overleaf that we like the most.

\section{Whitepaper Process with Overleaf}
Our process of using Overleaf includes several steps. After determining the topic and purpose of a new whitepaper, we open a new Overleaf document and post the link in Slack, our communication system. Multiple different team members then work together to outline the paper and then draft the different sections. Different people work on different sections at once.

In drafting the paper, we make extensive use of the fact that Overleaf is a LaTeX editor. We organize content by sections and subsections, helping break down the paper for readers. The table of contents is automatically generated, useful when we change sections around, which happens frequently. We use footnotes, text formatting (code, monospace, italics, bold, etc.), figure insertion, and the glossary function, to write our papers.

Both during drafting and afterwards, we edit and revise our paper. We use the ability to leave comments on text to communicate feedback. The person who wrote the section in question can come back and revise, or reply to the comments to discuss the issue. Others can see this and add their input.

When everyone editing the paper is satisfied, we publish the paper. We download a PDF from Overleaf, add it to Google Drive, and post the link to the Google Drive PDF on our website. We post the PDF on Chief Delphi, the FIRST Robotics Competition forum, where other members of the community can read and comment on the papers.

\section{Features and Shortcomings}
In the years we've been using Overleaf for documentation, we've grown to appreciate many of the features of the software. 

Since our robotics team regularly publishes at least one or two papers a year, we place a lot of emphasis on the continuity of our documentation. It makes it a lot easier to understand previous years' work when the formatting is consistent throughout years. Using LaTeX with Overleaf has allowed us to reference a LaTeX template to exactly replicate the formatting. It also allows our papers to have a clean, professional appearance without the writers having to spend a lot of time in formatting.

We also appreciate Overleaf's feature which allows us to have multiple editors at once, and to have editors make comments in the Overleaf document. We try to include everyone we can in the writing and editing of our yearly whitepapers, because summarizing the work of the year helps in understanding that work better. The ability to have seven different students and mentors reading, writing, and editing the paper in real time during robotics meetings makes this possible. 

When we first started using OverLeaf, we would mainly share PDFs of the documents for revision in Slack. We realized quickly that it was difficult to keep track of changes which we had and hadn't made and it was difficult to access previous versions. With Overleaf Premium, we're able to use the history feature to see previous versions of the document. The ability to view the revision history makes it possible to have many different editors without cluttering our Slack channels with ``WhitepaperFinal.pdf'', ``WhitepaperFinalFinal.pdf'', and ``WhitepaperFinalFinalv2.pdf'' eternally.

The only shortcoming we've run into in our use of the software is the fact that anyone who wants to view the document while we're editing it needs to create an Overleaf account. This provides a slight barrier for anyone who wants to skim the document for formatting or grammar, but doesn't regularly write or edit documents on Overleaf.

\section{Outcomes}
We've published a few papers using Overleaf, the most recent being our 2019 programming paper, an update on our activities with ROS (Robot Operating System). That whitepaper can be found \href{https://team900.org/blog/ZebROS1.1/}{here}. Our papers have been downloaded hundreds of times by members of our robotics community and beyond. They serve not only as references for people outside of the robotics team, but they serve as internal documentation of our progress each year. 

We would like to thank Overleaf for providing us with this software, which has allowed us to write and edit professional, consistent whitepapers for documentation and curriculum.

\end{document}
