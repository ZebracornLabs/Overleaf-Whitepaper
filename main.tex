\documentclass{zebracorns} 

\title{Zebracorn Template}
\author{The Zebracorns}
\date{Date Published}

\begin{document}

\maketitle

\section{Introduction}


\section{Whitepaper Process with Overleaf}
\subsection{editing, comments, multiple editors, drafting, sending versions in ROS, publishing, using features of LaTeX, }
Our process of using Overleaf includes several steps. After determining the topic and purpose of a new whitepaper, we open a new Overleaf document and post the link in Slack, our communication system. Multiple different team members then work together to outline the paper and then draft the different sections. Different people work on different sections at once.\\

In drafting the paper, we make extensive use of the fact that Overleaf is a LaTeX editor. We organize content by sections and subsections, helping break down the paper for readers. The table of contents is automatically generated, useful when we change sections around, which happens frequently. We use footnotes, text formatting (code, monospace, italics, bold, etc.), figure insertion, and the glossary function, to write our papers.\\

Both during drafting and afterwards, we edit and revise our paper. We use the ability to leave comments on text to communicate feedback. The person who wrote the section in question can come back and revise, or reply to the comments to discuss the issue. Others can see this and add their input.\\

When everyone editing the paper is satisfied, we publish the paper. We download a PDF from Overleaf, add it to Google Drive, and post the link to the Google Drive PDF on our website. We post the PDF on Chief Delphi, the FIRST Robotics Competition forum, where other members of the community can read and comment on the papers.\\

\section{Features and Shortcomings}
In the years we've been using Overleaf for documentation, we've grown to appreciate many of the features of the software. \\

Since our robotics team regularly publishes at least one or two papers a year, we place a lot of emphasis on the continuity of our documentation. It makes it a lot easier to understand previous years' work when the formatting is consistent throughout years. Using LaTeX with Overleaf has allowed us to reference a LaTeX template to exactly replicate the formatting. It also allows our papers to have a clean, professional appearance without the writers having to spend a lot of time in formatting.\\

We also appreciate Overleaf's feature which allows us to have multiple editors at once. We try to include everyone we can in the writing and editing of our yearly whitepapers, because summarizing the work of the year helps in understanding that work better. The ability to have seven different students and mentors reading, writing, and editing the paper in real time during robotics meetings makes this possible. \\

When we first started using OverLeaf, we would mainly share PDFs of the documents for revision in Slack. We realized quickly that it was difficult to keep track of changes which we had and hadn't made and it was difficult to access previous versions. Now, the main way we handle revisions is by encouraging editors to make comments in the Overleaf document. The ability to comment makes it possible to have many different editors without cluttering our Slack channels with "WhitepaperFinal.pdf", "WhitepaperFinalFinal.pdf", and "WhitepaperFinalFinalv2.pdf" eternally.\\

We've also run into a few shortcomings of our use in the software 

%%Finally, 
%
%Features
%-looks cool (professional appearance)
%-reference editing section - all that useful stuff
%    -comments
%    -multiple editors
%    -LaTeX (organization, formatting, consistency in formatting across years (template), glossary)
%
%shortcomings
% - don't have access to history
% - people have to make an account
% - 

\section{Outcomes}
We've published a few papers using Overleaf, the most recent being ZebROS 2019, an update on our activities with ROS (Robot Operating System) that can be found HERE.
 - Zebracorns have published this many papers
 - read by this many people
 - on this, this, and this topic

\end{document}
